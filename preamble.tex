%%%%%%%%%%%%%%%%%%%%%%%%%%%%%%%%%%%%%%%%%%%%%%%%%%%%%%%%%%%%
%                                                          %
% Author:       Léo RACLET                                 %
% File:         preamble.tex                               %
% Description:  LaTeX configuration file                   %
% License:      MIT                                        %
% Version:      v1.0                                       %
%                                                          %
%%%%%%%%%%%%%%%%%%%%%%%%%%%%%%%%%%%%%%%%%%%%%%%%%%%%%%%%%%%%

% ======================================================== %
% PACKAGES                                                 %
% ======================================================== %

\usepackage[utf8]{inputenc}              % Text encoding
\usepackage[T1]{fontenc}                 % Font encoding
\usepackage[french]{babel}               % French
\usepackage[framemethod=TikZ]{mdframed}  % Frames
\usepackage[ruled, vlined]{algorithm2e}  % Algorithms
\usepackage[Rejne]{fncychap}             % Fancy chapter

\usepackage{geometry}                    % Document layout
\usepackage{listings}                    % Code listing
\usepackage{fancyhdr}                    % Fancy header and footer
\usepackage{pgfornament}                 % Ornaments
\usepackage{tikz}                        % Draws
\usepackage{graphicx}                    % Figures
\usepackage{xcolor}                      % More colors
\usepackage{caption}                     % Captions
\usepackage{subcaption}                  % Sub-captions
\usepackage{hyperref}                    % Hyperlinks
\usepackage{amsmath}                     % Beautiful maths
\usepackage{amssymb}                     % Math symbols
\usepackage{biblatex}                    % Bibliography
\usepackage{pgfplots}                    % Plots
\usepackage{float}                       % Floating objects
\usepackage{mathtools}                   % Better math formatting
\usepackage{csquotes}                    % Quotes
\usepackage{enumitem}                    % Better lists
\usepackage{booktabs}                    % Quality tables
\usepackage{array}                       % Fixed tables columns width
\usepackage{multirow}                    % Combines tables rows
\usepackage{tabularx}                    % Fixed tables width
\usepackage{nicematrix}                  % Better matrices
\usepackage{etoc}                        % Mini tables of contents
\usepackage{multicol}                    % Mulitple columns
\usepackage{ragged2e}                    % Text alignment
\usepackage{subfiles}                    % Include subfiles

% Pgfplot package specific version
\pgfplotsset{compat=1.7}

% ======================================================== %
% TIKZ LIBRARIES                                           %
% ======================================================== %

\usetikzlibrary{shadows}
\usetikzlibrary{arrows.meta}
\usetikzlibrary{angles}
\usetikzlibrary{quotes}

% Optionals
\usetikzlibrary{calc}
\usetikzlibrary{arrows}
\usetikzlibrary{positioning}
\usetikzlibrary{calligraphy}
\usetikzlibrary{pgfplots.dateplot}
\usetikzlibrary{decorations.pathreplacing}
\usetikzlibrary{decorations.markings}
\usetikzlibrary{decorations.text}

% ======================================================== %
% LAYOUT                                                   %
% ======================================================== %

% Document layout
\geometry{
  paper=a4paper,
  margin=70pt
}

% ======================================================== %
% TEXT & LINKS                                             %
% ======================================================== %

% Hyperlinks style
\hypersetup{
  colorlinks,
  linkcolor=black
}

% ======================================================== %
% SPACING & INDENTATION                                    %
% ======================================================== %

\frenchspacing               % Remove extra spaces (caused by "babel" package)
\reversemarginpar            % Reset paragraph margin
\setlength{\parindent}{0pt}  % No indent before paragraph by default

% ======================================================== %
% CHAPTERS & TITLES                                        %
% ======================================================== %

% Chapter title
\newcommand{\parttitle}{}

% Numbered chapters
\newcommand{\chap}[2][0]{
  \chapter{#2}
  \renewcommand{\parttitle}{#2}
}

% Unnumbered chapters
\newcommand{\unbchap}[2][0]{
  \chapter*{#2}
  \addcontentsline{toc}{chapter}{#2}
  \renewcommand{\parttitle}{#2}
}

% ======================================================== %
% COLORS                                                   %
% ======================================================== %

% Code listing colors
\definecolor{codegreen}{rgb}{0,0.6,0}
\definecolor{codegray}{rgb}{0.5,0.5,0.5}
\definecolor{codepurple}{rgb}{0.58,0,0.82}
\definecolor{backcolour}{rgb}{0.95,0.95,0.92}

% ======================================================== %
% COUNTERS                                                 %
% ======================================================== %

\setcounter{secnumdepth}{6}  % Depth of numbered sections
\setcounter{tocdepth}{4}     % Depth of table of contents

% Sections numbering
\renewcommand{\thechapter}{\Roman{chapter}}
\renewcommand{\thesection}{\arabic{section}}
\renewcommand{\thesubsection}{\thesection.\arabic{subsection}}
\renewcommand{\thesubsubsection}{\thesubsection.\arabic{subsubsection}}
\renewcommand{\thefigure}{\thesection.\arabic{figure}}

% ======================================================== %
% CAPTIONS                                                 %
% ======================================================== %

\DeclareCaptionLabelSeparator{custom}{ -- }           % Caption label separator
\DeclareCaptionFormat{custom}{{\sc #1#2}{\small #3}}  % Caption format

\captionsetup{
  format=custom,
  labelsep=custom
}

% ======================================================== %
% HEADER & FOOTER                                          %
% ======================================================== %

% Header height :
% (15pt needed because of the "fancyhdr" package)

\setlength{\headheight}{15pt}

\pagestyle{fancy}
\fancyhead[R]{Léo Raclet}                  % Left: Current chapter number
\fancyhead[L]{Chapitre \arabic{chapter}}   % Right: Current chapter title
\fancyhead[C]{\sc \parttitle}              % Center: Author

% ======================================================== %
% CODE LISTING                                             %
% ======================================================== %

% Define comments
\SetKwComment{Comment}{/* }{ */}

% Code listing style
\lstdefinestyle{mystyle}{
  backgroundcolor=\color{backcolour!30},
  commentstyle=\color{codepurple},
  keywordstyle=\color{blue},
  numberstyle=\tiny\color{codegray},
  stringstyle=\color{codepurple},
  basicstyle=\ttfamily\footnotesize\bfseries,
  breakatwhitespace=false,
  breaklines=true,
  captionpos=t,
  keepspaces=true,
  numbers=left,
  numbersep=12pt,
  showspaces=false,
  showstringspaces=false,
  showtabs=false,
  tabsize=2
}

% Configure code listing
\lstset{
  style=mystyle,
  framexleftmargin=24pt,
  frame=shadowbox,
  rulesepcolor=\color{black},
  linewidth=\linewidth,
  xleftmargin=24pt,
  aboveskip=12pt,
  belowskip=12pt
}

% ======================================================== %
% MDFRAME ENVIRONMENTS                                     %
% ======================================================== %

% -------------------------------------------------------- %
% Environments for theorems
% -------------------------------------------------------- %

\newenvironment{theorem}[2][]
{
  \mdfsetup{
    frametitle={
      \tikz[]
      \node[anchor=east,rectangle,rounded corners=2pt,fill=blue!20, draw=blue!50]
      {Théorème: \sc #1};
    },
    roundcorner=5pt,
    shadow=true,
    innertopmargin=10pt,
    linecolor=blue!20,
    linewidth=1pt,
    topline=true,
    backgroundcolor=blue!2,
    frametitleaboveskip=\dimexpr-\ht\strutbox\relax
  }
  \begin{mdframed}
    \relax
    \label{#2}
  }
  {
  \end{mdframed}
}

% -------------------------------------------------------- %
% Environments for proofs
% -------------------------------------------------------- %

\newenvironment{proof}[2][]
{
  \mdfsetup{
    frametitle={
      \tikz[]
      \node[anchor=east,rectangle,rounded corners=2pt,fill=red!20, draw=red!50]
      {Preuve: \sc #1};
    },
    roundcorner=5pt,
    shadow=true,
    innertopmargin=10pt,
    linecolor=red!20,
    linewidth=1pt,
    topline=true,
    backgroundcolor=red!2,
    frametitleaboveskip=\dimexpr-\ht\strutbox\relax
  }
  \begin{mdframed}
    \relax
    \label{#2}
  }
  {
  \end{mdframed}
}

% -------------------------------------------------------- %
% Environments for lemmas
% -------------------------------------------------------- %

\newenvironment{lemma}[2][]
{
  \mdfsetup{
    frametitle={
      \tikz[]
      \node[anchor=east,rectangle,rounded corners=2pt,fill=green!20, draw=green!50]
      {Lemme: \sc #1};
    },
    roundcorner=5pt,
    shadow=true,
    innertopmargin=10pt,
    linecolor=green!40,
    linewidth=1pt,
    topline=true,
    backgroundcolor=green!2,
    frametitleaboveskip=\dimexpr-\ht\strutbox\relax
  }
  \begin{mdframed}
    \relax
    \label{#2}
  }
  {
  \end{mdframed}
}

% -------------------------------------------------------- %
% Environments for definitions
% -------------------------------------------------------- %

\newenvironment{definition}[2][]
{
  \mdfsetup{
    frametitle={
      \tikz[]
      \node[anchor=east]
      {\textcolor{black}{Définition :}};
    },
    shadowcolor=black,
    shadow=true,
    innertopmargin=0pt,
    linecolor=black!60,
    linewidth=1pt,
    topline=true,
    backgroundcolor=black!2,
  }
  \begin{mdframed}
    \relax
    \label{#2}
  }
  {
  \end{mdframed}
}

% -------------------------------------------------------- %
% Environments for examples
% -------------------------------------------------------- %

\newenvironment{example}[2][]
{
  \mdfsetup{
    frametitle={
      \tikz[]
      \node[anchor=east]
      {\textcolor{red!20!orange}{\sc Exemple :}};
    },
    innertopmargin=0pt,
    linecolor=red!20!orange,
    linewidth=1pt,
    topline=false,
    bottomline=false,
  }
  \begin{mdframed}
    \relax
    \label{#2}
  }
  {
  \end{mdframed}
}

% -------------------------------------------------------- %
% Environments for remarks
% -------------------------------------------------------- %

\newenvironment{remark}[2][]
{
  \mdfsetup{
    frametitle={
      \tikz[]
      \node[anchor=east]
      {\textcolor{violet}{\sc Remarque :}};
    },
    innertopmargin=0pt,
    linecolor=violet,
    linewidth=1pt,
    topline=false,
    bottomline=false,
  }
  \begin{mdframed}
    \relax
    \label{#2}
  }
  {
  \end{mdframed}
}

% -------------------------------------------------------- %
% Environments for properties
% -------------------------------------------------------- %

\newenvironment{property}[2][]
{
  \mdfsetup{
    frametitle={
      \tikz[]
      \node[anchor=east]
      {\textcolor{red}{Propriété :}};
    },
    innertopmargin=0pt,
    linecolor=red,
    linewidth=1pt,
    topline=false,
    bottomline=false,
  }
  \begin{mdframed}
    \relax
    \label{#2}
  }
  {
  \end{mdframed}
}

% -------------------------------------------------------- %
% Environments for important notes
% -------------------------------------------------------- %

\newenvironment{important}[2][]
{
  \mdfsetup{
    frametitle={
      \tikz[]
      \node[anchor=east,rectangle,fill=red!20, draw=red!50]
      {\textcolor{black}{Important !}};
    },
    shadowcolor=red,
    shadow=true,
    innertopmargin=10pt,
    linecolor=red!60,
    linewidth=1pt,
    topline=true,
    backgroundcolor=red!2,
    frametitleaboveskip=\dimexpr-\ht\strutbox\relax
  }
  \begin{mdframed}
    \relax
    \label{#2}
  }
  {
  \end{mdframed}
}

% -------------------------------------------------------- %
% Environments for exercises
% -------------------------------------------------------- %

\newenvironment{exercise}[2][]
{
  \mdfsetup{
    frametitle={
      \tikz[]
      \node[anchor=east]
      {\textcolor{black!80}{Exercice #1: }};
    },
    innertopmargin=0pt,
    linecolor=black,
    innerlinewidth=0.4pt,
    outerlinewidth=0.4pt,
    middlelinecolor=black!5,
    linewidth=2pt,
    topline=false,
    rightline=false,
    bottomline=false,
  }
  \begin{mdframed}
    \relax
    \label{#2}
  }
  {
  \end{mdframed}
}

% -------------------------------------------------------- %
% Environments for contours
% -------------------------------------------------------- %

\newenvironment{contour}[2][]
{
  \mdfsetup{
    shadowsize=4pt,
    shadowcolor=black,
    shadow=true,
    innertopmargin=10pt,
    linecolor=black!60,
    linewidth=1pt,
    topline=true,
    backgroundcolor=black!1,
  }
  \begin{mdframed}
    \relax
    \label{#2}
  }
  {
  \end{mdframed}
}

% -------------------------------------------------------- %
% Environments for syntheses
% -------------------------------------------------------- %

\newenvironment{synthese}[2][]
{
  \mdfsetup{
    frametitle={#1},
    frametitlealignment=\centering,
    innertopmargin=10pt,
    linewidth=0.5pt,
    frametitlerule=true,
    frametitlerulewidth=0.5pt,
    frametitlebackgroundcolor=yellow!3,
    frametitleaboveskip=10pt,
    frametitlebelowskip=10pt,
    backgroundcolor=black!1,
    roundcorner=10pt,
  }
  \begin{mdframed}
    \relax
    \label{#2}
  }
  {
  \end{mdframed}
}

% ======================================================== %
% FIGURES, DRAWS & PLOTS                                   %
% ======================================================== %

% Tikz drawing
\newenvironment{draw}[2][]
{
  \begin{figure}[ht]
    \centering
    \caption{#2}
    \begin{tikzpicture}[#1]
    }
    {
    \end{tikzpicture}
  \end{figure}
}

%%%%%%%%%%%%%%%%%%%%%%%%%%%%%%%%%%%%%%%%%%%%%%%%%%%%%%%%%%%%
%%%%%%%%%%%%%%%%%%%%%%%%%%%%%%%%%%%%%%%%%%%%%%%%%%%%%%%%%%%%