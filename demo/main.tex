%%%%%%%%%%%%%%%%%%%%%%%%%%%%%%%%%%%%%%%%%%%%%%%%%%%%%%%%%%%%
%
% Author:       Léo RACLET
% File:         main.tex
% Description:  LaTeX main document
% License:      MIT
% Version:      v1.0
%
%%%%%%%%%%%%%%%%%%%%%%%%%%%%%%%%%%%%%%%%%%%%%%%%%%%%%%%%%%%%

\documentclass[12pt, french]{report}
%%%%%%%%%%%%%%%%%%%%%%%%%%%%%%%%%%%%%%%%%%%%%%%%%%%%%%%%%%%%
%                                                          %
% Author:       Léo RACLET                                 %
% File:         preamble.tex                               %
% Description:  LaTeX configuration file                   %
% License:      MIT                                        %
% Version:      v1.0                                       %
%                                                          %
%%%%%%%%%%%%%%%%%%%%%%%%%%%%%%%%%%%%%%%%%%%%%%%%%%%%%%%%%%%%

% ======================================================== %
% PACKAGES                                                 %
% ======================================================== %

\usepackage[utf8]{inputenc}              % Text encoding
\usepackage[T1]{fontenc}                 % Font encoding
\usepackage[french]{babel}               % French
\usepackage[framemethod=TikZ]{mdframed}  % Frames
\usepackage[ruled, vlined]{algorithm2e}  % Algorithms
\usepackage[Rejne]{fncychap}             % Fancy chapter

\usepackage{geometry}                    % Document layout
\usepackage{listings}                    % Code listing
\usepackage{fancyhdr}                    % Fancy header and footer
\usepackage{pgfornament}                 % Ornaments
\usepackage{tikz}                        % Draws
\usepackage{graphicx}                    % Figures
\usepackage{xcolor}                      % More colors
\usepackage{caption}                     % Captions
\usepackage{subcaption}                  % Sub-captions
\usepackage{hyperref}                    % Hyperlinks
\usepackage{amsmath}                     % Beautiful maths
\usepackage{amssymb}                     % Math symbols
\usepackage{biblatex}                    % Bibliography
\usepackage{pgfplots}                    % Plots
\usepackage{float}                       % Floating objects
\usepackage{mathtools}                   % Better math formatting
\usepackage{csquotes}                    % Quotes
\usepackage{enumitem}                    % Better lists
\usepackage{booktabs}                    % Quality tables
\usepackage{array}                       % Fixed tables columns width
\usepackage{multirow}                    % Combines tables rows
\usepackage{tabularx}                    % Fixed tables width
\usepackage{nicematrix}                  % Better matrices
\usepackage{etoc}                        % Mini tables of contents
\usepackage{multicol}                    % Mulitple columns
\usepackage{ragged2e}                    % Text alignment
\usepackage{subfiles}                    % Include subfiles

% Pgfplot package specific version
\pgfplotsset{compat=1.7}

% ======================================================== %
% TIKZ LIBRARIES                                           %
% ======================================================== %

\usetikzlibrary{shadows}
\usetikzlibrary{arrows.meta}
\usetikzlibrary{angles}
\usetikzlibrary{quotes}

% Optionals
\usetikzlibrary{calc}
\usetikzlibrary{arrows}
\usetikzlibrary{positioning}
\usetikzlibrary{calligraphy}
\usetikzlibrary{pgfplots.dateplot}
\usetikzlibrary{decorations.pathreplacing}
\usetikzlibrary{decorations.markings}
\usetikzlibrary{decorations.text}

% ======================================================== %
% LAYOUT                                                   %
% ======================================================== %

% Document layout
\geometry{
  paper=a4paper,
  margin=70pt
}

% ======================================================== %
% TEXT & LINKS                                             %
% ======================================================== %

% Hyperlinks style
\hypersetup{
  colorlinks,
  linkcolor=black
}

% ======================================================== %
% SPACING & INDENTATION                                    %
% ======================================================== %

\frenchspacing               % Remove extra spaces (caused by "babel" package)
\reversemarginpar            % Reset paragraph margin
\setlength{\parindent}{0pt}  % No indent before paragraph by default

% ======================================================== %
% CHAPTERS & TITLES                                        %
% ======================================================== %

% Chapter title
\newcommand{\parttitle}{}

% Numbered chapters
\newcommand{\chap}[2][0]{
  \chapter{#2}
  \renewcommand{\parttitle}{#2}
}

% Unnumbered chapters
\newcommand{\unbchap}[2][0]{
  \chapter*{#2}
  \addcontentsline{toc}{chapter}{#2}
  \renewcommand{\parttitle}{#2}
}

% ======================================================== %
% COLORS                                                   %
% ======================================================== %

% Code listing colors
\definecolor{codegreen}{rgb}{0,0.6,0}
\definecolor{codegray}{rgb}{0.5,0.5,0.5}
\definecolor{codepurple}{rgb}{0.58,0,0.82}
\definecolor{backcolour}{rgb}{0.95,0.95,0.92}

% ======================================================== %
% COUNTERS                                                 %
% ======================================================== %

\setcounter{secnumdepth}{6}  % Depth of numbered sections
\setcounter{tocdepth}{4}     % Depth of table of contents

% Sections numbering
\renewcommand{\thechapter}{\Roman{chapter}}
\renewcommand{\thesection}{\arabic{section}}
\renewcommand{\thesubsection}{\thesection.\arabic{subsection}}
\renewcommand{\thesubsubsection}{\thesubsection.\arabic{subsubsection}}
\renewcommand{\thefigure}{\thesection.\arabic{figure}}

% ======================================================== %
% CAPTIONS                                                 %
% ======================================================== %

\DeclareCaptionLabelSeparator{custom}{ -- }           % Caption label separator
\DeclareCaptionFormat{custom}{{\sc #1#2}{\small #3}}  % Caption format

\captionsetup{
  format=custom,
  labelsep=custom
}

% ======================================================== %
% HEADER & FOOTER                                          %
% ======================================================== %

% Header height :
% (15pt needed because of the "fancyhdr" package)

\setlength{\headheight}{15pt}

\pagestyle{fancy}
\fancyhead[R]{Léo Raclet}                  % Left: Current chapter number
\fancyhead[L]{Chapitre \arabic{chapter}}   % Right: Current chapter title
\fancyhead[C]{\sc \parttitle}              % Center: Author

% ======================================================== %
% CODE LISTING                                             %
% ======================================================== %

% Define comments
\SetKwComment{Comment}{/* }{ */}

% Code listing style
\lstdefinestyle{mystyle}{
  backgroundcolor=\color{backcolour!30},
  commentstyle=\color{codepurple},
  keywordstyle=\color{blue},
  numberstyle=\tiny\color{codegray},
  stringstyle=\color{codepurple},
  basicstyle=\ttfamily\footnotesize\bfseries,
  breakatwhitespace=false,
  breaklines=true,
  captionpos=t,
  keepspaces=true,
  numbers=left,
  numbersep=12pt,
  showspaces=false,
  showstringspaces=false,
  showtabs=false,
  tabsize=2
}

% Configure code listing
\lstset{
  style=mystyle,
  framexleftmargin=24pt,
  frame=shadowbox,
  rulesepcolor=\color{black},
  linewidth=\linewidth,
  xleftmargin=24pt,
  aboveskip=12pt,
  belowskip=12pt
}

% ======================================================== %
% MDFRAME ENVIRONMENTS                                     %
% ======================================================== %

% -------------------------------------------------------- %
% Environments for theorems
% -------------------------------------------------------- %

\newenvironment{theorem}[2][]
{
  \mdfsetup{
    frametitle={
      \tikz[]
      \node[anchor=east,rectangle,rounded corners=2pt,fill=blue!20, draw=blue!50]
      {Théorème: \sc #1};
    },
    roundcorner=5pt,
    shadow=true,
    innertopmargin=10pt,
    linecolor=blue!20,
    linewidth=1pt,
    topline=true,
    backgroundcolor=blue!2,
    frametitleaboveskip=\dimexpr-\ht\strutbox\relax
  }
  \begin{mdframed}
    \relax
    \label{#2}
  }
  {
  \end{mdframed}
}

% -------------------------------------------------------- %
% Environments for proofs
% -------------------------------------------------------- %

\newenvironment{proof}[2][]
{
  \mdfsetup{
    frametitle={
      \tikz[]
      \node[anchor=east,rectangle,rounded corners=2pt,fill=red!20, draw=red!50]
      {Preuve: \sc #1};
    },
    roundcorner=5pt,
    shadow=true,
    innertopmargin=10pt,
    linecolor=red!20,
    linewidth=1pt,
    topline=true,
    backgroundcolor=red!2,
    frametitleaboveskip=\dimexpr-\ht\strutbox\relax
  }
  \begin{mdframed}
    \relax
    \label{#2}
  }
  {
  \end{mdframed}
}

% -------------------------------------------------------- %
% Environments for lemmas
% -------------------------------------------------------- %

\newenvironment{lemma}[2][]
{
  \mdfsetup{
    frametitle={
      \tikz[]
      \node[anchor=east,rectangle,rounded corners=2pt,fill=green!20, draw=green!50]
      {Lemme: \sc #1};
    },
    roundcorner=5pt,
    shadow=true,
    innertopmargin=10pt,
    linecolor=green!40,
    linewidth=1pt,
    topline=true,
    backgroundcolor=green!2,
    frametitleaboveskip=\dimexpr-\ht\strutbox\relax
  }
  \begin{mdframed}
    \relax
    \label{#2}
  }
  {
  \end{mdframed}
}

% -------------------------------------------------------- %
% Environments for definitions
% -------------------------------------------------------- %

\newenvironment{definition}[2][]
{
  \mdfsetup{
    frametitle={
      \tikz[]
      \node[anchor=east]
      {\textcolor{black}{Définition :}};
    },
    shadowcolor=black,
    shadow=true,
    innertopmargin=0pt,
    linecolor=black!60,
    linewidth=1pt,
    topline=true,
    backgroundcolor=black!2,
  }
  \begin{mdframed}
    \relax
    \label{#2}
  }
  {
  \end{mdframed}
}

% -------------------------------------------------------- %
% Environments for examples
% -------------------------------------------------------- %

\newenvironment{example}[2][]
{
  \mdfsetup{
    frametitle={
      \tikz[]
      \node[anchor=east]
      {\textcolor{red!20!orange}{\sc Exemple :}};
    },
    innertopmargin=0pt,
    linecolor=red!20!orange,
    linewidth=1pt,
    topline=false,
    bottomline=false,
  }
  \begin{mdframed}
    \relax
    \label{#2}
  }
  {
  \end{mdframed}
}

% -------------------------------------------------------- %
% Environments for remarks
% -------------------------------------------------------- %

\newenvironment{remark}[2][]
{
  \mdfsetup{
    frametitle={
      \tikz[]
      \node[anchor=east]
      {\textcolor{violet}{\sc Remarque :}};
    },
    innertopmargin=0pt,
    linecolor=violet,
    linewidth=1pt,
    topline=false,
    bottomline=false,
  }
  \begin{mdframed}
    \relax
    \label{#2}
  }
  {
  \end{mdframed}
}

% -------------------------------------------------------- %
% Environments for properties
% -------------------------------------------------------- %

\newenvironment{property}[2][]
{
  \mdfsetup{
    frametitle={
      \tikz[]
      \node[anchor=east]
      {\textcolor{red}{Propriété :}};
    },
    innertopmargin=0pt,
    linecolor=red,
    linewidth=1pt,
    topline=false,
    bottomline=false,
  }
  \begin{mdframed}
    \relax
    \label{#2}
  }
  {
  \end{mdframed}
}

% -------------------------------------------------------- %
% Environments for important notes
% -------------------------------------------------------- %

\newenvironment{important}[2][]
{
  \mdfsetup{
    frametitle={
      \tikz[]
      \node[anchor=east,rectangle,fill=red!20, draw=red!50]
      {\textcolor{black}{Important !}};
    },
    shadowcolor=red,
    shadow=true,
    innertopmargin=10pt,
    linecolor=red!60,
    linewidth=1pt,
    topline=true,
    backgroundcolor=red!2,
    frametitleaboveskip=\dimexpr-\ht\strutbox\relax
  }
  \begin{mdframed}
    \relax
    \label{#2}
  }
  {
  \end{mdframed}
}

% -------------------------------------------------------- %
% Environments for exercises
% -------------------------------------------------------- %

\newenvironment{exercise}[2][]
{
  \mdfsetup{
    frametitle={
      \tikz[]
      \node[anchor=east]
      {\textcolor{black!80}{Exercice #1: }};
    },
    innertopmargin=0pt,
    linecolor=black,
    innerlinewidth=0.4pt,
    outerlinewidth=0.4pt,
    middlelinecolor=black!5,
    linewidth=2pt,
    topline=false,
    rightline=false,
    bottomline=false,
  }
  \begin{mdframed}
    \relax
    \label{#2}
  }
  {
  \end{mdframed}
}

% -------------------------------------------------------- %
% Environments for contours
% -------------------------------------------------------- %

\newenvironment{contour}[2][]
{
  \mdfsetup{
    shadowsize=4pt,
    shadowcolor=black,
    shadow=true,
    innertopmargin=10pt,
    linecolor=black!60,
    linewidth=1pt,
    topline=true,
    backgroundcolor=black!1,
  }
  \begin{mdframed}
    \relax
    \label{#2}
  }
  {
  \end{mdframed}
}

% -------------------------------------------------------- %
% Environments for syntheses
% -------------------------------------------------------- %

\newenvironment{synthese}[2][]
{
  \mdfsetup{
    frametitle={#1},
    frametitlealignment=\centering,
    innertopmargin=10pt,
    linewidth=0.5pt,
    frametitlerule=true,
    frametitlerulewidth=0.5pt,
    frametitlebackgroundcolor=yellow!3,
    frametitleaboveskip=10pt,
    frametitlebelowskip=10pt,
    backgroundcolor=black!1,
    roundcorner=10pt,
  }
  \begin{mdframed}
    \relax
    \label{#2}
  }
  {
  \end{mdframed}
}

% ======================================================== %
% FIGURES, DRAWS & PLOTS                                   %
% ======================================================== %

% Tikz drawing
\newenvironment{draw}[2][]
{
  \begin{figure}[ht]
    \centering
    \caption{#2}
    \begin{tikzpicture}[#1]
    }
    {
    \end{tikzpicture}
  \end{figure}
}

%%%%%%%%%%%%%%%%%%%%%%%%%%%%%%%%%%%%%%%%%%%%%%%%%%%%%%%%%%%%
%%%%%%%%%%%%%%%%%%%%%%%%%%%%%%%%%%%%%%%%%%%%%%%%%%%%%%%%%%%%

% ======================================================== %
% DOCUMENT                                                 %
% ======================================================== %

\begin{document}

\begin{titlepage}
  \begin{center}
    \vspace{10cm}
    {\Large \itshape 2023-2024}\\
    \vspace{3cm}
    \pgfornament[width=10cm]{88}\\
    \vspace{2mm}
    \vspace{0.5cm}
    {\huge Mathématiques}\\
    \vspace{0.5cm}
    {\huge FISE 1}\\
    \vspace{0.5cm}
    \pgfornament[width=10cm]{88}\\
    \vfill
    Léo {\sc Raclet}\\
  \end{center}
\end{titlepage}

\tableofcontents
\addcontentsline{toc}{chapter}{Table des matières}
\listoffigures
\addcontentsline{toc}{chapter}{Table des figures}
\listoftables
\addcontentsline{toc}{chapter}{Liste des tableaux}

\etocsettocstyle{
  \section*{Sommaire}
  \hrule
  \bigskip
  \begin{minipage}{.95\linewidth}
  }
  {
  \end{minipage}
  \bigskip
  \hrule
}

\unbchap{Avant-propos}
\chap{Les fondamentaux}
\localtableofcontents

\section{Mathématiques}

\subsection{Blocs de couleurs}

\begin{theorem}[Théorème de Pythagore (oui)]{thm:pythagoras}
  In a right triangle, the square of the hypotenuse is equal to the sum of the squares of the catheti.
  $$a^2+b^2=c^2$$
\end{theorem}
In mathematics, the Pythagorean theorem, also known as Pythagoras' theorem (see theorem \ref{thm:pythagoras}), is a relation in Euclidean geometry among the three sides of a right triangle.

\begin{proof}{proof:pythagoras}
  If $x=y=\sqrt{2}$ is an example, then we are done; otherwise $\sqrt{2}^{\sqrt{2}}$ is irrational, in which case taking $x=\sqrt{2}^{\sqrt{2}}$ and $y=\sqrt{2}$ gives us:
  \[\bigg(\sqrt{2}^{\sqrt{2}}\bigg)^{\sqrt{2}}=\sqrt{2}^{\sqrt{2}\sqrt{2}}=\sqrt{2}^{2}=2.\]
\end{proof}

\newpage

\begin{lemma}[Identité de Bézout]{lemma:bezout}
  Let $a$ and $b$ be nonzero integers and let $d$ be their greatest common divisor. Then there exist integers $x$ and $y$ such that:
  \[ax+by=d\]
\end{lemma}

\begin{definition}[]{def:one}
  If we want dummy text in our document then we generally search for lorem ipsum text generators and copy-paste those texts/paragraphs in our document.
\end{definition}

\begin{example}[]{exaple:one}
  In LaTeX, we don't need to do such copy and paste thing. LaTeX has different packages which automatically generates dummy text in our document. You can generate them with just a few lines of code.
\end{example}

\begin{remark}[]{remark:one}
  In the example below, lipsum package is used to print dummy text below chapter title. lipsum[2-4] prints lorem ipsum text from paragraph 2 to paragraph 4
\end{remark}

\begin{property}[]{property:one}
  Every line in your source code must end with otherwise your algorithm will continue on the same line of text in the output.
\end{property}

\begin{lstlisting}[language=Python]
# libraries:
import numpy as np
import pandas as pd
import matplotlib.pyplot as plt  # creating the variables:
x = np.linspace(0,100)
y = x**2  # plotting
plt.plot(x, y, '-b')
\end{lstlisting}

\begin{algorithm}
  \caption{An algorithm with caption}
  \KwData{$n \geq 0$}
  \KwResult{$y = x^n$}
  $y \gets 1$\;
  $X \gets x$\;
  $N \gets n$\;
  \While{$N \neq 0$}{
    \eIf{$N$ is even}{
      $X \gets X \times X$\;
      $N \gets \frac{N}{2}$ \Comment*[r]{This is a comment}
    }{\If{$N$ is odd}{
        $y \gets y \times X$\;
        $N \gets N - 1$\;
      }
    }
  }
\end{algorithm}

\begin{important}[]{imp:one}
  If we want dummy text in our document then we generally search for lorem ipsum text generators and copy-paste those texts/paragraphs in our document.
\end{important}

\newpage

\begin{exercise}[]{ex:one}
  If we want dummy text in our document then we generally search for lorem ipsum text generators and copy-paste those texts/paragraphs in our document.
\end{exercise}

\begin{contour}[]{con:one}
  If we want dummy text in our document then we generally search for lorem ipsum text generators and copy-paste those texts/paragraphs in our document.
\end{contour}

\begin{multicols}{2}
  \begin{synthese}[Rappel]{syn:one}
    If we want dummy text in our document then we generally search for lorem ipsum text generators and copy-paste those texts/paragraphs in our document.
  \end{synthese}
  \begin{synthese}[Ordre bien fondé]{syn:two}
    Un ordre est bien fondé s'il n'existe pas
    de suite infiniment strictement décrois-
    sante, i.e. toute partie non vide de E ad-
    met un élément minimal.
    \smallskip

    Un ordre est bien fondé s'il n'existe pas
    de suite infiniment strictement décrois-
    sante, i.e. toute partie non vide de E ad-
    met un élément minimal.
  \end{synthese}
\end{multicols}

\begin{figure}[ht]
  \centering
  \begin{tikzpicture}
    \begin{scope}[every node/.style={circle,thick,draw}]
      \node (A) at (0,0) {A};
      \node (B) at (0,3) {B};
      \node (C) at (2.5,4) {C};
      \node (D) at (2.5,1) {D};
      \node (E) at (2.5,-3) {E};
      \node (F) at (5,3) {F} ;
    \end{scope}
    \begin{scope}[>={Stealth[black]},
        every node/.style={fill=white,circle},
      every edge/.style={draw=red,very thick}]
      \path [->] (A) edge node {$5$} (B);
      \path [->] (B) edge node {$3$} (C);
      \path [->] (A) edge node {$4$} (D);
      \path [->] (D) edge node {$3$} (C);
      \path [->] (A) edge node {$3$} (E);
      \path [->] (D) edge node {$3$} (E);
      \path [->] (D) edge node {$3$} (F);
      \path [->] (C) edge node {$5$} (F);
      \path [->] (E) edge node {$8$} (F);
      \path [->] (B) edge[bend right=60] node {$1$} (E);
    \end{scope}
  \end{tikzpicture}
  \caption{This is a real beautiful figure of a graph}
\end{figure}

\begin{figure}[ht]
  \centering
  \begin{tikzpicture}[domain=0:4]
    \draw[very thin,color=gray] (-0.1,-1.1) grid (3.9,3.9);

    \draw[->] (-0.2,0) -- (4.2,0) node[right] {$x$};
    \draw[->] (0,-1.2) -- (0,4.2) node[above] {$f(x)$};

    \draw[color=red]    plot (\x,\x)             node[right] {$f(x) =x$};
    % \x r means to convert '\x' from degrees to _r_adians:
    \draw[color=blue]   plot (\x,{sin(\x r)})    node[right] {$f(x) = \sin x$};
    \draw[color=orange] plot (\x,{0.05*exp(\x)}) node[right] {$f(x) = \frac{1}{20} \mathrm e^x$};
  \end{tikzpicture}
\end{figure}

\begin{draw}[my angle/.style={draw, ->, angle eccentricity=1.3, angle radius=9mm}]{Trigonometric circle}
  % coordinate axis
  \draw[-] (-2.0,0) -- (2.0,0);
  \draw[-] (0,-2.0) -- (0,2.0);
  % circle
  \draw (0,0) circle (2cm);
  % coordinates
  \coordinate[label=right:{$( 1,0)$}] (A)  at ( 2,0);
  \coordinate[label=above:{$(0, 1)$}] (B)  at ( 0,2);
  \coordinate[label=left:{$(-1,0)$}] (C)  at (-2,0);
  \coordinate[label=below:{$(0,-1)$}] (D)  at (0,-2);
  %
  \coordinate[label=above:M]  (M) at (60:2);
  \coordinate                 (O) at ( 0:0);
  % angles
  \draw[thick]    (M) -- (C)  (M) -- (O);
  \pic[my angle, "$t$"]      {angle = A--C--M};
  \pic[my angle, "$\theta$"] {angle = A--O--M};
\end{draw}

\clearpage
\end{document}

%%%%%%%%%%%%%%%%%%%%%%%%%%%%%%%%%%%%%%%%%%%%%%%%%%%%%%%%%%%%
%%%%%%%%%%%%%%%%%%%%%%%%%%%%%%%%%%%%%%%%%%%%%%%%%%%%%%%%%%%%